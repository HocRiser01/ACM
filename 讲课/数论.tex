\documentclass{beamer}
\usepackage{ctex, hyperref}
\usepackage[T1]{fontenc}
\usepackage{latexsym,amsmath,xcolor,multicol,booktabs,calligra}
\usepackage{graphicx,pstricks,listings,stackengine}
\def\cmd#1{\texttt{\color{red}\footnotesize $\backslash$#1}}
\def\env#1{\texttt{\color{blue}\footnotesize #1}}
\definecolor{deepblue}{rgb}{0,0,0.5}
\definecolor{deepred}{rgb}{0.6,0,0}
\definecolor{deepgreen}{rgb}{0,0.5,0}
\definecolor{halfgray}{gray}{0.55}
\lstset{
	basicstyle=\ttfamily\small,
	keywordstyle=\bfseries\color{deepblue},
	emphstyle=\ttfamily\color{deepred},
	stringstyle=\color{deepgreen},
	numbers=left,
	numberstyle=\small\color{halfgray},
	rulesepcolor=\color{red!20!green!20!blue!20},
	frame=shadowbox,
}

%---------------------------------------------------

\author{HocRiser}
\title{数论进阶}
\institute{吉林大学 20级唐计}
\date{2021年1月27日}
\usepackage{waseda}

\begin{document}

\kaishu
\begin{frame}
	\titlepage
	\begin{figure}[htpb]
		\begin{center}
			\includegraphics[width=0.2\linewidth]{pic/HocRiser.jpg}
		\end{center}
	\end{figure}
\end{frame}

\begin{frame}
	\tableofcontents[sectionstyle=show,subsectionstyle=show/shaded/hide,subsubsectionstyle=show/shaded/hide]
\end{frame}

%---------------------------------------------------

\section{预备知识}

\subsection{素数计数}

\begin{frame}
\begin{itemize}[<+-| alert@+>]
	\item 令素数计数函数$\pi(n)$表示不超过$n$的素数个数。我们有如下的素数定理:$$\pi(n) \sim \frac{n}{\ln n}$$
	\item 推论:$n$附近的素数密度近似是$\frac{1}{\ln n}$
	\item 第$n$个素数$p_n \sim n\ln n$
\end{itemize}
\end{frame}

\subsection{数论函数与积性函数}
\begin{frame}
\begin{itemize}[<+-| alert@+>]
	\item 定义域为正整数、值域是复数的子集的函数称为数论函数。
	\item 设$f$是数论函数,若$\forall a,b \in \mathbb{N*}$且$a \perp b$,$f(ab)=f(a)f(b)$,则称 $f$是积性函数。
	\item 若$\forall a,b \in \mathbb{N*}$,都有$f(ab)=f(a)f(b)$,则称$f$是完全
积性的。
\end{itemize}
\end{frame}

\begin{frame}
\begin{itemize}[<+-| alert@+>]
	\item 若$f(n)$是积性函数,且$n=p_{1}^{\alpha_1}p_{2}^{\alpha_2} \cdots p_{s}^{\alpha_s}$是$n$的标准分解,则有$$f(n)=f(p_{1}^{\alpha_1})f(p_{2}^{\alpha_2}) \cdots f(p_{s}^{\alpha_s})$$
	\item 因此积性函数$f$可以转化为研究$f(p^\alpha)$,即$f$在素数和素数的幂上的取值。
	\item 对于完全积性函数,往往只需研究$f$在素数上的取值。
\end{itemize}
\end{frame}

\begin{frame}
\begin{itemize}[<+-| alert@+>]
	\item 若$f(n)$是积性函数,且$n=p_{1}^{\alpha_1}p_{2}^{\alpha_2} \cdots p_{s}^{\alpha_s}$是$n$的标准分解,则有$$f(n)=f(p_{1}^{\alpha_1})f(p_{2}^{\alpha_2}) \cdots f(p_{s}^{\alpha_s})$$
	\item 因此积性函数$f$可以转化为研究$f(p^\alpha)$,即$f$在素数和素数的幂上的取值。
	\item 对于完全积性函数,往往只需研究$f$在素数上的取值。
	\item 对于$n$以内$f$函数的计算,可以在$Euler$筛法的过程中线性得到结果。
\end{itemize}
\end{frame}

\subsection{常见积性函数}

\begin{frame}
\begin{itemize}[<+-| alert@+>]
	\item 单位函数$\epsilon(n)$
	\item $$\epsilon(n)=[n=1]=
			\begin{cases}
			1,n=1\\
			0,n \neq 1
			\end{cases}$$
	\item 除数函数$\sigma_{k}(n)=\sum_{d|n}d^{k}$,$\sigma_{0}(n)$ 常记作 $d(n)$,约数和$\sigma_{1}(n)$常记作$\sigma(n)$。
	\item 幂函数$Id_{k}(n)=n^{k}$,$Id=Id_{1}$
\end{itemize}
\end{frame}

\begin{frame}
\begin{itemize}[<+-| alert@+>]
	\item 欧拉函数$\varphi(n)$ 表示不超过 $n$ 且与 $n$ 互质的正整数的个数
		$$n=\sum_{d\mid n}\varphi(d)$$
	\item 莫比乌斯函数$\mu$,
		$$\mu(n)=\begin{cases}
				1,&n=1\\
				(-1)^{s},&n=p_{1}p_{2}\dots p_{s}\\
				0,&\text{otherwise}
			\end{cases}$$
		其中 $p_{1},\dots,p_{s}$  是不同素数。
\end{itemize}
\end{frame}

\section{积性函数优化算法}

\subsection{Dirichlet卷积}

\begin{frame}
\begin{itemize}[<+-| alert@+>]
	\item 设 $f$ , $g$ 是数论函数,考虑数论函数 $h$ 满足$$h(n)=\sum_{d\mid n}f(d)g(\frac{n}{d})$$
	\item 则称 $h$ 为 $f$ 和 $g$ 的Dirichlet卷积,记作 $h=f*g$
\end{itemize}
\end{frame}

\begin{frame}
\begin{itemize}[<+-| alert@+>]
	\item 单位函数 $\epsilon$ 是Dirichlet卷积的单位元,即对于任意函数 $f$ ,有 $\epsilon * f=f*\epsilon=f$
	\item Dirichlet卷积满足交换律、结合律和分配律
		$$f*g=g*f$$
		$$(f*g)*h=f*(g*h)$$
		$$(f+g)*h=f*h+g*h$$
		$$(xf)*g=x(f*g)$$
	\item 如果 $f$ , $g$ 都是积性函数,那么 $f*g$ 也是积性函数
	\item 设 $f\cdot g(x)=f(x)\times g(x)$ ,$f$ 是完全积性函数, $g,h$ 是数论函数,则 $(f\cdot g)*(f\cdot h)=f\cdot (g*h)$
\end{itemize}
\end{frame}

\begin{frame}
\begin{itemize}[<+-| alert@+>]
	\item 用 $1$ 表示取值恒为 $1$ 的常函数,则除数函数的定义可以写为$\sigma_{k}=1*Id_{k}$
	\item 欧拉函数的性质可以写为$Id=\varphi * 1$
	\item 莫比乌斯函数的性质可以写为$\epsilon=\mu * 1$
\end{itemize}
\end{frame}

\begin{frame}
\begin{itemize}[<+-| alert@+>]
	\item 欧拉函数性质证明
		\begin{itemize}[<+-| alert@+>]
			\item 将 $1$ 到 $n$ 之间的所有正整数 $i$ 按照与 $n$ 的最大公约数 $d=\gcd(i,n)$ 分类,我们分别统计 $d$ 相同的类中 $i$ 的个数:考虑 $\gcd(\frac{i}{d},\frac{n}{d})=1$ ,我们统计了满足这个条件的 $\frac{i}{d}$ 的个数就等价于统计了原类中 $i$ 的个数,而这样的 $\frac{i}{d}$ 实际上就是小于等于 $\frac{n}{d}$ 且与 $\frac{n}{d}$ 互质的数的个数,就是 $\varphi(\frac{n}{d})$ ,而这里的 $\frac{n}{d}$ 又与 $\sum_{d\mid n}\varphi(d)$ 中的 $d$ 是一一对应的,我们对每一类的个数求和写成式子就是 $\sum_{d\mid n}\varphi(d)$ ,这每个数都会被统计恰好1次 ,总和正好是 $n$ 
		\end{itemize}
	\item 莫比乌斯函数性质证明:
		\begin{itemize}[<+-| alert@+>]
			\item $n=1$ 时显然成立,如果 $n>1$ ,设 $n$ 有 $s$ 个不同的素因子,由于 $\mu(d)\neq 0$ 当且仅当 $d$ 无平方因子,故 $d$ 中每个素因子的指数只能为 $0$ 或 $1$ 才又贡献。故有$$\sum_{d\mid n} \mu(d)=\sum_{k=0}^{s}(-1)^{k}\binom{s}{k}=(1-1)^{s}=0$$
		\end{itemize}
\end{itemize}
\end{frame}

\subsection{Mobius反演}

\begin{frame}
\begin{itemize}[<+-| alert@+>]
	\item 莫比乌斯变换:
	\item 设 $f$ 是数论函数,定义函数 $g$ 满足$$g(n)=\sum_{d\mid n}f(d)$$
		则称 $g$ 是 $f$ 的Mobius变换,$f$ 是 $g$ 的Mobius逆变换
	\item 用Dirichlet卷积表示即为 $g=f*1$
\end{itemize}
\end{frame}

\begin{frame}
\begin{itemize}[<+-| alert@+>]
	\item Mobius反演定理指出,Mobius变换的充要条件是$$f(n)=\sum_{d\mid n}g(d)\mu(\frac{n}{d})$$
	\item 即 $g=f*1\Leftrightarrow f=g*\mu$
	\item 证明可以使用Dirichlet卷积:$$g=f*1\Leftrightarrow f=f*\epsilon=f*1*\mu=g*\mu$$
\end{itemize}
\end{frame}

\begin{frame}
	$$\mu*1=\epsilon$$
	$$\sigma_{k}=Id_{k}*1$$
	$$Id=\varphi * 1$$
	$$\varphi=\mu*Id$$
	$$d(ij)=\sum_{x\mid i}\sum_{y\mid j}[\gcd(x,y)=1]$$
	$$\sum_{d\mid n}\frac{\mu(d)}{d}=\frac{\varphi(n)}{n}$$
\end{frame}

\begin{frame}
	$$\mu(ab)=\mu(a)\mu(b)[a\perp b]$$
	$$\varphi(n)=\sum_{i=1}^{n}[\gcd(i,n)=1]=\sum_{i=1}^{n}\sum_{k\mid i,k\mid n}\mu(k)=\sum_{k\mid n}\mu(k)\lfloor\frac{n}{k}\rfloor$$
	$$\sum_{i=1}^{n}i\times[\gcd(i,n)=1]=\begin{cases}1,n=1\\ \frac{n\times \varphi(n)}{2},n>1\end{cases}$$
	$$\sum_{d\mid n}\sum_{i=1}^{d}i\times [\gcd(i,d)=1]=\frac{1}{2}(1+\sum_{d\mid n}d\times \varphi(d))$$
	$$\gcd(a,b)=\sum_{d\mid\gcd(a,b)}\varphi(d)=\sum_{d\mid a,d\mid b}\varphi(d)$$
\end{frame}

\section{杜教筛}

\subsection{莫比乌斯函数前缀和}

\begin{frame}
\begin{itemize}[<+-| alert@+>]
	\item 求$f(n)=\sum_{i=1}^{n}\mu(i)$
	$$\sum_{k=1}^{n}(\mu*1)(k)=\sum_{i=1}^{n}1(i)\sum_{j=1}^{\lfloor\frac{n}{i}\rfloor}\mu(j)=\sum_{i=1}^{n}f(\lfloor\frac{n}{i}\rfloor)$$
	$$\sum_{k=1}^{n}(\mu*1)(k)=\sum_{k=1}^{n}\epsilon(k)=1$$
	$$f(n)=1-\sum_{i=2}^{n}f(\lfloor\frac{n}{i}\rfloor)$$
	\item 时间复杂度:$$T(n)=\sum_{i=1}^{n}O(\sqrt{\lfloor\frac{n}{i}\rfloor})=\sum_{i=1}^{\sqrt{n}}O(\sqrt{\lfloor\frac{n}{i}\rfloor})+\sum_{i=1}^{\sqrt{n}}O(i)\approx O(n^{0.75})$$
\end{itemize}
\end{frame}

\subsection{欧拉函数前缀和}

\begin{frame}
\begin{itemize}[<+-| alert@+>]
	\item 求$f(n)=\sum_{i=1}^{n}\varphi(i)$
	$$\sum_{k=1}^{n}(\varphi*1)(k)=\sum_{i=1}^{n}1(i)\sum_{j=1}^{\lfloor\frac{n}{i}\rfloor}\varphi(j)=\sum_{i=1}^{n}f(\lfloor\frac{n}{i}\rfloor)$$
	$$\sum_{k=1}^{n}(\varphi*1)(k)=\sum_{k=1}^{n}Id(k)=\frac{n(n+1)}{2}$$
	$$f(n)=\frac{n(n+1)}{2}-\sum_{i=2}^{n}f(\lfloor\frac{n}{i}\rfloor)$$
\end{itemize}
\end{frame}

\begin{frame}
\begin{itemize}[<+-| alert@+>]
	\item 杜教筛的本质是求$F=\sum f$,但$F$很难计算,于是尝试构造$g$使得$f*g=h$,满足$G$和$H$很好计算
	\item 有等式$\sum_{k=1}^{n}(f*g)(k)=\sum_{i=1}^{n}g(i)\sum_{j=1}^{\lfloor\frac{n}{i}\rfloor}f(j)=\sum_{i=1}^{n}g(i)F(\lfloor\frac{n}{i}\rfloor)$
	\item 又有$\sum_{k=1}^{n}(f*g)(k)=\sum_{k=1}^{n}h(k)=H(n)$
	\item 移项有$F(n)=H(n)-\sum_{i=2}^{n}F(\lfloor\frac{n}{i}\rfloor)$
\end{itemize}
\end{frame}

\subsection{变形}

\begin{frame}
\begin{itemize}[<+-| alert@+>]
	\item 求$f(k)=\mu(k)*k$,$F(n)=\sum_{i=1}^{n}f(i)$
	\item $$(f*Id)(n)=\sum_{d|n}f(d)*Id(\frac{n}{d})=\sum_{d|n}\mu(d)*d*\frac{n}{d}=n*\sum_{d|n}\mu(d)=\epsilon(n)$$
	\item 即得到$f*Id=\epsilon$,之后可以得到$F(n)=1-\sum_{i=2}^{n}i*F(\lfloor\frac{n}{i}\rfloor)$
\end{itemize}
\end{frame}

\begin{frame}
\begin{itemize}[<+-| alert@+>]
	\item 求$\sum\varphi(i)*i$,$\sum\mu(i)*i^2$与上面类似
	\item 本质是对于$f(k)=f_0(k)*h(k)$,其中$h$为完全积性函数
	\item 令$g(k)=r(k)*h(k)$,那么$(f*g)(k)=(f_{0}*r)(k)*h(k)$,如果$G$和$\sum(f*g)$很好计算,就可以类比之前的方法
\end{itemize}
\end{frame}

%---------------------------------------------------

\end{document}